\documentclass{article}
\usepackage[portuguese]{babel}
\usepackage{textcomp,xcolor}

%%%%%%%%%% Start TeXmacs macros
\catcode`\>=\active \def>{
\fontencoding{T1}\selectfont\symbol{62}\fontencoding{\encodingdefault}}
\newcommand{\tmcolor}[2]{{\color{#1}{#2}}}
\newcommand{\nonconverted}[1]{\mbox{}}
%%%%%%%%%% End TeXmacs macros

\begin{document}

\title{Hypotenuse and Hyperbolic}

\maketitle

\tmcolor{black}{\

The terms "hypotenuse" and "hyperbolic" are related only by their shared
etymological root in the Greek word "hypo," meaning "under" or "beneath."
However, in modern mathematics, they refer to completely different concepts.
Let's break them down:

\ }

\section{**Hypotenuse:**}

\

\tmcolor{black}{* \ \ **Definition:** The hypotenuse is the **longest side**
of a **right-angled triangle**, and it is always the side **opposite the right
angle** (90-degree angle).

\

* \ \ **Context:** It's a fundamental part of Euclidean geometry and is most
famously associated with the Pythagorean theorem (a{\texttwosuperior} +
b{\texttwosuperior} = c{\texttwosuperior}, where c is the hypotenuse).

\

* \ \ **Example:** Imagine a triangle with angles of 90 degrees, 60 degrees,
and 30 degrees. \ The side opposite the 90-degree angle is the hypotenuse.

\

\section{**Hyperbolic:**}

\

* \ \ **Definition:** "Hyperbolic" refers to things related to
**hyperbolas**, which are a type of conic section (a curve formed by the
intersection of a plane and a double cone). Hyperbolic functions are analogs
of trigonometric functions but defined using the hyperbola instead of the
circle.

\

* \ \ **Context:** It's a concept in analytic geometry and calculus. It
involves hyperbolic functions (sinh, cosh, tanh, coth, sech, csch), hyperbolic
geometry (a non-Euclidean geometry where parallel lines diverge), and
hyperbolic equations (differential equations).

\

* \ \ **Example:** The equation *y* = cosh(*x*) represents a hyperbolic
cosine function, which has a graph shaped like a catenary curve (a hanging
chain). \ Hyperbolic geometry is used in physics, particularly in the theory
of relativity.

\

\section{**Shared Root, Different Meanings:**}

\

The Greek prefix "hypo" ("under") is present in both:

\

* \ \ **Hypotenuse:**

Comes from the Greek "hypoteinousa," which means "stretching under." \ This
refers to the side that stretches under the right angle to connect the other
two sides.

\

* \ \ **Hyperbola:** Derives from the Greek "hyperbole," meaning
"overthrowing" or "exceeding." \ This refers to how a hyperbola "exceeds" or
opens more widely than a parabola. (Although the exact connection is not
immediately obvious from the shape of a hyperbola. Appolonius originally named
them, as an extension of the way he named ellipses and parabolas.)

\

**In summary:** While the words share a common origin, they have evolved to
represent distinct and unrelated concepts in mathematics. One deals with the
sides of a right-angled triangle, and the other deals with curves and
functions based on a hyperbola. It's a case of etymology leading to different
destinations.

\ }

\title{Hypotenuse about Hyperbolic Cosh and Sinh Mathematics}

\maketitle

\tmcolor{black}{Here's a thinking process to arrive at the explanation
connecting the hypotenuse (from right triangles) to hyperbolic cosine (cosh)
and hyperbolic sine (sinh):

\

1. \ **Identify the Core Concepts:** The request asks about the relationship
between "hypotenuse" and "hyperbolic cosh/sinh." I know:

\ \ \ * \ \ **Hypotenuse:** Longest side of a right triangle, opposite the
90{\textdegree} angle. Related to the Pythagorean theorem (a{\texttwosuperior}
+ b{\texttwosuperior} = c{\texttwosuperior}). Connected to standard
trigonometric functions (sin, cos, tan) via SOH CAH TOA, where sin({\theta}) =
opposite/hypotenuse, cos({\theta}) = adjacent/hypotenuse.

\ \ \ * \ \ **Hyperbolic Functions (cosh, sinh):** Defined using the
exponential function: sinh(x) = (e\^{}x - e\^{}-x)/2, cosh(x) = (e\^{}x +
e\^{}-x)/2. They are *analogs* of trigonometric functions. They relate to the
*unit hyperbola* (x{\texttwosuperior} - y{\texttwosuperior} = 1), not the unit
circle (x{\texttwosuperior} + y{\texttwosuperior} = 1).

\

2. \ **Initial Question:** Is there a *direct* geometric relationship like
SOH CAH TOA where cosh or sinh directly involve the hypotenuse of a *standard*
right triangle?

\ \ \ * \ \ *Answer:* No, not in the standard Euclidean geometry sense. Cosh
and sinh are defined algebraically using exponentials or geometrically using
the *hyperbola*. The hypotenuse belongs to right triangles in Euclidean space,
linked to the *circle* (via Pythagorean theorem and trig functions).

\

3. \ **Look for Analogies:** Since cosh and sinh are *analogs* of cos and
sin, what's the core relationship for cos and sin?

\ \ \ * \ \ The fundamental identity: cos{\texttwosuperior}({\theta}) +
sin{\texttwosuperior}({\theta}) = 1.

\ \ \ * \ \ This comes from the Pythagorean theorem applied to a right
triangle inscribed in a unit circle. If the hypotenuse is 1, the adjacent side
is cos({\theta}) and the opposite side is sin({\theta}). So,
(adjacent){\texttwosuperior} + (opposite){\texttwosuperior} =
(hypotenuse){\texttwosuperior}, which becomes cos{\texttwosuperior}({\theta})
+ sin{\texttwosuperior}({\theta}) = 1{\texttwosuperior}.

\

4. \ **What's the Analogous Identity for Hyperbolic Functions?**

\ \ \ * \ \ Recall or derive the hyperbolic identity:
cosh{\texttwosuperior}(x) - sinh{\texttwosuperior}(x) = 1.

\ \ \ * \ \ Derivation:

\ \ \ \ \ \ \ * \ \ cosh{\texttwosuperior}(x) = [(e\^{}x +
e\^{}-x)/2]{\texttwosuperior} = (e\^{}2x + 2e\^{}x e\^{}-x + e\^{}-2x)/4 =
(e\^{}2x + 2 + e\^{}-2x)/4

\ \ \ \ \ \ \ * \ \ sinh{\texttwosuperior}(x) = [(e\^{}x -
e\^{}-x)/2]{\texttwosuperior} = (e\^{}2x - 2e\^{}x e\^{}-x + e\^{}-2x)/4 =
(e\^{}2x - 2 + e\^{}-2x)/4

\ \ \ \ \ \ \ * \ \ cosh{\texttwosuperior}(x) - sinh{\texttwosuperior}(x) =
[(e\^{}2x + 2 + e\^{}-2x) - (e\^{}2x - 2 + e\^{}-2x)] / 4 = (2 - (-2)) / 4 =
4/4 = 1.

\

5. \ **Connect the Identities:**

\ \ \ * \ \ Circular Trig: cos{\texttwosuperior}({\theta}) +
sin{\texttwosuperior}({\theta}) = 1 (related to x{\texttwosuperior} +
y{\texttwosuperior} = 1, the unit circle)

\ \ \ * \ \ Hyperbolic Trig: cosh{\texttwosuperior}(x) -
sinh{\texttwosuperior}(x) = 1 (related to x{\texttwosuperior} -
y{\texttwosuperior} = 1, the unit hyperbola)

\

6. \ **Geometric Interpretation:**

\ \ \ * \ \ **Circle:** A point on the unit circle (x, y) can be
parameterized as x = cos({\theta}), y = sin({\theta}). Here, {\theta} is
related to the angle and the arc length. The distance from the origin
(radius/hypotenuse of an implied triangle) is 1.

\ \ \ * \ \ **Hyperbola:** A point on the *right branch* of the unit
hyperbola (x, y) can be parameterized as x = cosh(a), y = sinh(a). Here, *'a'*
is *not* a standard angle but is related to the *area* of a hyperbolic sector.
The coordinates (cosh(a), sinh(a)) satisfy the hyperbola equation
x{\texttwosuperior} - y{\texttwosuperior} = 1.

\

7. \ **Can we form a triangle involving cosh and sinh?**

\ \ \ * \ \ Consider the point P(cosh(a), sinh(a)) on the hyperbola
x{\texttwosuperior} - y{\texttwosuperior} = 1.

\ \ \ * \ \ Consider the origin O(0, 0).

\ \ \ * \ \ Consider the point X(cosh(a), 0) on the x-axis.

\ \ \ * \ \ We have a right triangle OXP.

\ \ \ \ \ \ \ * \ \ Side OX (adjacent) has length cosh(a).

\ \ \ \ \ \ \ * \ \ Side XP (opposite) has length sinh(a).

\ \ \ \ \ \ \ * \ \ What about the side OP (the distance from the origin to
the point on the hyperbola)? This would be the hypotenuse of triangle OXP.

\ \ \ * \ \ Using the Pythagorean theorem for *this specific triangle OXP*:

\ \ \ \ \ \ \ * \ \ (Hypotenuse OP){\texttwosuperior} = (Adjacent
OX){\texttwosuperior} + (Opposite XP){\texttwosuperior}

\ \ \ \ \ \ \ * \ \ (Hypotenuse OP){\texttwosuperior} =
cosh{\texttwosuperior}(a) + sinh{\texttwosuperior}(a)

\ \ \ \ \ \ \ * \ \ *Wait!* This is *not* the fundamental identity. The
fundamental identity is cosh{\texttwosuperior}(a) - sinh{\texttwosuperior}(a)
= 1.

\

8. \ **Reconciling the Triangle and the Identity:**

\ \ \ * \ \ The triangle OXP exists in standard Euclidean space. Its sides
are indeed cosh(a), sinh(a), and  \nonconverted{sqrt}
(cosh{\texttwosuperior}(a) + sinh{\texttwosuperior}(a)).

\ \ \ * \ \ The identity cosh{\texttwosuperior}(a) -
sinh{\texttwosuperior}(a) = 1 comes from the definition of the *hyperbola*
itself (x{\texttwosuperior} - y{\texttwosuperior} = 1) where x = cosh(a) and y
= sinh(a).

\ \ \ * \ \ So, while cosh(a) and sinh(a) can be the lengths of the legs of a
Euclidean right triangle, the *hypotenuse* of that triangle is 
\nonconverted{sqrt} (cosh{\texttwosuperior}(a) + sinh{\texttwosuperior}(a)),
which is *not* equal to 1 (unlike the circular case where the hypotenuse *is*
1 for the unit circle).

\

9. \ **Synthesize the Explanation:**

\ \ \ * \ \ Start by stating the *lack* of direct relationship in the SOH CAH
TOA sense.

\ \ \ * \ \ Explain the *analogous* nature: sin/cos relate to the circle
(x{\texttwosuperior}+y{\texttwosuperior}=1), sinh/cosh relate to the hyperbola
(x{\texttwosuperior}-y{\texttwosuperior}=1).

\ \ \ * \ \ Bring in the fundamental identities:
cos{\texttwosuperior}({\theta})+sin{\texttwosuperior}({\theta})=1 vs.
cosh{\texttwosuperior}(a)-sinh{\texttwosuperior}(a)=1.

\ \ \ * \ \ Explain the unit circle parametrization: (cos {\theta}, sin
{\theta}). The hypotenuse of the triangle formed by (0,0), (cos {\theta}, 0),
(cos {\theta}, sin {\theta}) is 1.

\ \ \ * \ \ Explain the unit hyperbola parametrization: (cosh a, sinh a).

\ \ \ * \ \ Form the Euclidean right triangle with vertices (0,0), (cosh a,
0), and (cosh a, sinh a).

\ \ \ * \ \ Identify the legs: The legs have lengths \textbar cosh a\textbar 
and \textbar sinh a\textbar . Since cosh(a) is always {\geq} 1 for real 'a',
and we usually consider the right branch, the lengths are cosh(a) and \textbar
sinh(a)\textbar .

\ \ \ * \ \ Calculate the hypotenuse of *this* triangle using the Pythagorean
theorem: Hypotenuse{\texttwosuperior} = cosh{\texttwosuperior}(a) +
sinh{\texttwosuperior}(a).

\ \ \ * \ \ **Crucial Point:** This hypotenuse is *not* generally equal to 1.
It is related to cosh{\texttwosuperior}(a) + sinh{\texttwosuperior}(a), while
the defining identity involves cosh{\texttwosuperior}(a) *minus*
sinh{\texttwosuperior}(a).

\ \ \ * \ \ Mention the connection via the identity:
cosh{\texttwosuperior}(a) + sinh{\texttwosuperior}(a) = cosh(2a). So the
hypotenuse length is  \nonconverted{sqrt} (cosh(2a)).

\ \ \ * \ \ Conclude by reiterating that the connection is *analogous* and
relates to the fundamental identities and parameterizations, but cosh and sinh
don't define ratios involving the hypotenuse in the same direct way as cos and
sin do for a *general* right triangle. The triangle involving cosh(a) and
sinh(a) as legs is a specific construction related to the hyperbola's
parameterization.}

Okay, let's clarify the relationship (or lack thereof) and the analogies
between the hypotenuse (from right triangles and standard trigonometry) and
the hyperbolic functions cosh and sinh.

\

**Direct Relationship? No.**

\

* \ \ The **hypotenuse** is a geometric concept specific to **right-angled
triangles** in Euclidean geometry.

* \ \ Standard trigonometric functions (sine, cosine) are defined using ratios
involving the sides of a right triangle (SOH CAH TOA):

\ \ \ * \ \ sin({\theta}) = Opposite / Hypotenuse

\ \ \ * \ \ cos({\theta}) = Adjacent / Hypotenuse

* \ \ **Hyperbolic functions (sinh, cosh)** are defined algebraically using
the exponential function *e* or geometrically using the **unit hyperbola**
(x{\texttwosuperior} - y{\texttwosuperior} = 1), *not* a right triangle in the
same direct way.

\ \ \ * \ \ cosh(x) = (eˣ + e⁻ˣ) / 2

\ \ \ * \ \ sinh(x) = (eˣ - e⁻ˣ) / 2

\

**The Analogous Relationship (The Key Connection):**

\

The connection comes from the **fundamental identities** that mirror each
other:

\

1. \ **Standard Trigonometry (Unit Circle: x{\texttwosuperior} +
y{\texttwosuperior} = 1):**

\ \ \ * \ \ The fundamental identity is: **cos{\texttwosuperior}({\theta}) +
sin{\texttwosuperior}({\theta}) = 1**

\ \ \ * \ \ Geometric interpretation: If you have a point (x, y) on the unit
circle (radius 1), then x = cos({\theta}) and y = sin({\theta}). The distance
from the origin to the point is the radius, which acts like the hypotenuse of
an implied right triangle with legs x and y. The Pythagorean theorem
(x{\texttwosuperior} + y{\texttwosuperior} = hypotenuse{\texttwosuperior})
becomes cos{\texttwosuperior}({\theta}) + sin{\texttwosuperior}({\theta}) =
1{\texttwosuperior}.

\

2. \ **Hyperbolic Functions (Unit Hyperbola: x{\texttwosuperior} -
y{\texttwosuperior} = 1):**

\ \ \ * \ \ The fundamental identity is: **cosh{\texttwosuperior}(a) -
sinh{\texttwosuperior}(a) = 1**

\ \ \ * \ \ Geometric interpretation: If you have a point (x, y) on the right
branch of the unit hyperbola (x{\texttwosuperior} - y{\texttwosuperior} = 1),
then x = cosh(a) and y = sinh(a). Here, 'a' is related to the *area* of a
hyperbolic sector, not a simple angle like {\theta}.

\

**Forming a Triangle with cosh and sinh:**

\

While sinh and cosh aren't defined *by* a hypotenuse ratio, we *can*
construct a right triangle using them:

\

* \ \ Consider the point P on the unit hyperbola: P = (cosh(a), sinh(a)).

* \ \ Consider the origin O = (0, 0).

* \ \ Consider the point X on the x-axis: X = (cosh(a), 0).

\

Now, look at the right-angled triangle **OXP**:

\

* \ \ The length of the adjacent side (OX) is **cosh(a)**.

* \ \ The length of the opposite side (XP) is **\textbar sinh(a)\textbar **.

* \ \ What is the length of the **hypotenuse (OP)** of *this specific
triangle*? We use the standard Pythagorean theorem:

\ \ \ * \ \ Hypotenuse{\texttwosuperior} = (Adjacent){\texttwosuperior} +
(Opposite){\texttwosuperior}

\ \ \ * \ \ Hypotenuse{\texttwosuperior} = cosh{\texttwosuperior}(a) +
sinh{\texttwosuperior}(a)

\ \ \ * \ \ Hypotenuse = ** \nonconverted{sqrt} (cosh{\texttwosuperior}(a) +
sinh{\texttwosuperior}(a))**

\

**Important Distinction:**

\

* \ \ Notice that the hypotenuse of this triangle OXP is  \nonconverted{sqrt}
(cosh{\texttwosuperior}(a) + sinh{\texttwosuperior}(a)).

* \ \ This is **NOT** equal to 1 (unlike the unit circle case).

* \ \ The identity related to the *hyperbola itself* is
cosh{\texttwosuperior}(a) - sinh{\texttwosuperior}(a) = 1.

\

*(As a side note: using another identity, cosh(2a) =
cosh{\texttwosuperior}(a) + sinh{\texttwosuperior}(a), the hypotenuse length
of triangle OXP is actually  \nonconverted{sqrt} (cosh(2a)).)*

\

**In Summary:**

\

* \ \ There's no direct definition of cosh or sinh using a hypotenuse ratio
like SOH CAH TOA.

* \ \ The connection is through the **analogous fundamental identities**:
cos{\texttwosuperior}({\theta}) + sin{\texttwosuperior}({\theta}) = 1 (circle)
vs. cosh{\texttwosuperior}(a) - sinh{\texttwosuperior}(a) = 1 (hyperbola).

* \ \ You *can* form a Euclidean right triangle where the legs have lengths
related to cosh(a) and sinh(a) based on the coordinates of a point on the
hyperbola. The hypotenuse of *that* specific triangle is  \nonconverted{sqrt}
(cosh{\texttwosuperior}(a) + sinh{\texttwosuperior}(a)), which is linked to
the identity cosh{\texttwosuperior}(a) + sinh{\texttwosuperior}(a) = cosh(2a),
not the defining identity cosh{\texttwosuperior}(a) -
sinh{\texttwosuperior}(a) = 1.

\

\title{hypotenuse and hyperbolic in quest of cosh and sinh to solver
equation}

\maketitle

Here's a thinking process to construct the explanation about using cosh and
sinh to solve equations, connecting back to the ideas of hypotenuse and
hyperbolic functions:

\

1. \ **Deconstruct the Request:** The user wants to understand how
"hypotenuse" and "hyperbolic" concepts, specifically related to \`{}cosh\`{}
and \`{}sinh\`{}, are used to *solve equations*.

\

2. \ **Recall Previous Context:** We've established:

\ \ \ * \ \ Hypotenuse belongs to right triangles (Euclidean geometry).

\ \ \ * \ \ Hyperbolic (cosh, sinh) relate to the hyperbola
x{\texttwosuperior} - y{\texttwosuperior} = 1 and have the identity
cosh{\texttwosuperior}(a) - sinh{\texttwosuperior}(a) = 1.

\ \ \ * \ \ There's no *direct* definition of cosh/sinh using a hypotenuse
ratio.

\ \ \ * \ \ The connection is *analogous* identities and the geometry of the
unit circle vs. the unit hyperbola.

\

3. \ **Identify the Core Problem:** How do \`{}cosh\`{} and \`{}sinh\`{} help
*solve equations*? This usually involves substitutions that simplify the
equation, leveraging their properties.

\

4. \ **Brainstorm Equation Types:** Where might \`{}cosh\`{} and \`{}sinh\`{}
substitutions be useful?

\ \ \ * \ \ Equations involving square roots of the form  \nonconverted{sqrt}
(x{\texttwosuperior} - 1) or  \nonconverted{sqrt} (x{\texttwosuperior} + 1).

\ \ \ * \ \ Integrals involving these forms.

\ \ \ * \ \ Differential equations (like the catenary).

\ \ \ * \ \ Systems of equations where the variables might naturally
represent cosh(a) and sinh(a).

\ \ \ * \ \ Equations that look structurally similar to trigonometric
identities, but with a minus sign.

\

5. \ **Focus on the Most Common Use Case (Algebra/Calculus):** The most
direct link to the "hypotenuse" analogy (even though it's imperfect) is
through substitutions simplifying expressions involving sums or differences of
squares, reminiscent of the Pythagorean theorem and its hyperbolic
counterpart.

\

\ \ \ * \ \ **Case 1: Expressions like  \nonconverted{sqrt}
(x{\texttwosuperior} - 1)**

\ \ \ \ \ \ \ * \ \ This looks like the hyperbolic identity:
cosh{\texttwosuperior}(a) - 1 = sinh{\texttwosuperior}(a).

\ \ \ \ \ \ \ * \ \ *Idea:* If we let x = cosh(a), then  \nonconverted{sqrt}
(x{\texttwosuperior} - 1) becomes  \nonconverted{sqrt}
(cosh{\texttwosuperior}(a) - 1) =  \nonconverted{sqrt}
(sinh{\texttwosuperior}(a)) = \textbar sinh(a)\textbar . This removes the
square root.

\ \ \ \ \ \ \ * \ \ *Constraint:* For x = cosh(a) to be valid, we need x
{\geq} 1 (since cosh(a) {\geq} 1 for all real 'a').

\

\ \ \ * \ \ **Case 2: Expressions like  \nonconverted{sqrt}
(x{\texttwosuperior} + 1)**

\ \ \ \ \ \ \ * \ \ This looks like rearranging the *other* hyperbolic
identity: sinh{\texttwosuperior}(a) + 1 = cosh{\texttwosuperior}(a).

\ \ \ \ \ \ \ * \ \ *Idea:* If we let x = sinh(a), then  \nonconverted{sqrt}
(x{\texttwosuperior} + 1) becomes  \nonconverted{sqrt}
(sinh{\texttwosuperior}(a) + 1) =  \nonconverted{sqrt}
(cosh{\texttwosuperior}(a)) = cosh(a) (since cosh(a) is always positive). This
removes the square root.

\ \ \ \ \ \ \ * \ \ *Constraint:* x = sinh(a) can represent any real number.

\

\ \ \ * \ \ **Contrast with Trigonometric Substitution (Hypotenuse Link):**

\ \ \ \ \ \ \ * \ \ For  \nonconverted{sqrt} (1 - x{\texttwosuperior}), we
use x = sin({\theta}) or x = cos({\theta}), leveraging
cos{\texttwosuperior}({\theta}) + sin{\texttwosuperior}({\theta}) = 1. This
directly relates to a right triangle with hypotenuse 1.

\ \ \ \ \ \ \ * \ \ For  \nonconverted{sqrt} (1 + x{\texttwosuperior}), we
use x = tan({\theta}), leveraging 1 + tan{\texttwosuperior}({\theta}) =
sec{\texttwosuperior}({\theta}). This relates to a right triangle with
adjacent side 1, opposite side x, and hypotenuse  \nonconverted{sqrt} (1 +
x{\texttwosuperior}).

\ \ \ \ \ \ \ * \ \ For  \nonconverted{sqrt} (x{\texttwosuperior} - 1), we
use x = sec({\theta}), leveraging sec{\texttwosuperior}({\theta}) - 1 =
tan{\texttwosuperior}({\theta}). This relates to a right triangle with
hypotenuse x, adjacent side 1, and opposite side  \nonconverted{sqrt}
(x{\texttwosuperior} - 1).

\

\ \ \ * \ \ **Connecting Back to "Hypotenuse":** While the *hyperbolic*
substitution itself doesn't *directly* use a hypotenuse ratio in its
*definition*, the *reason* it works is analogous to how trigonometric
substitutions work, which *are* derived from right triangles and the
Pythagorean theorem (involving a hypotenuse). The goal is the same: simplify
expressions involving sums/differences of squares using a fundamental
identity. The hyperbolic identity cosh{\texttwosuperior}(a) -
sinh{\texttwosuperior}(a) = 1 plays the role that
cos{\texttwosuperior}({\theta}) + sin{\texttwosuperior}({\theta}) = 1 plays in
trig substitutions.

\

6. \ **Structure the Explanation:**

\ \ \ * \ \ Start by acknowledging the user's quest -- using cosh/sinh to
solve equations.

\ \ \ * \ \ Reiterate the *indirect* connection to the hypotenuse concept
(via the analogous identity structure).

\ \ \ * \ \ State the primary technique: **Hyperbolic Substitution**.

\ \ \ * \ \ Explain the two main cases (\nonconverted{sqrt}
(x{\texttwosuperior} - 1) and  \nonconverted{sqrt} (x{\texttwosuperior} + 1)),
detailing the substitution, the simplification using the identity, and any
constraints.

\ \ \ * \ \ Provide a concrete example equation for each case.

\ \ \ * \ \ Briefly mention other areas (integrals, differential equations)
where these functions appear.

\ \ \ * \ \ Summarize: The power lies in the algebraic identity
cosh{\texttwosuperior}(a) - sinh{\texttwosuperior}(a) = 1 (and related forms),
which mirrors the Pythagorean identity used in trigonometric substitutions
(which *is* directly related to the hypotenuse).

\

7. \ **Refine and Clarify:**

\ \ \ * \ \ Ensure the distinction between the *definition* of cosh/sinh and
their *use* in substitutions is clear.

\ \ \ * \ \ Emphasize that the "hypotenuse" connection is more about the
*structural analogy* of the identities than a direct geometric application in
the hyperbolic case.

\ \ \ * \ \ Use clear formatting (bolding, bullet points) to break down the
information.

\ \ \ * \ \ Check that the examples are simple enough to illustrate the
point. Make sure to solve *for the original variable* in the examples.

\

8. \ **Self-Correction/Review:** Did I directly answer the user's prompt
about solving equations? Yes. Did I link it back to the "hypotenuse" and
"hyperbolic" concepts as requested? Yes, by explaining the analogous
identities and contrasting with trig substitution. Is the explanation clear
and accurate? Seems so.

\

Okay, let's connect the concepts. While the **hypotenuse** itself isn't
directly *used* in the definition or fundamental properties of **hyperbolic
functions (cosh, sinh)**, the *structure* of the core hyperbolic identity
allows us to use these functions to solve certain types of equations, often
those involving square roots that resemble the Pythagorean theorem (which
*does* involve the hypotenuse).

\

The key is the fundamental identity:

\

**cosh{\texttwosuperior}(a) - sinh{\texttwosuperior}(a) = 1**

\

This identity is the hyperbolic analogue of the trigonometric (Pythagorean)
identity:

\

**cos{\texttwosuperior}({\theta}) + sin{\texttwosuperior}({\theta}) = 1**
(This one *is* directly related to a right triangle with hypotenuse 1).

\

**How cosh and sinh Help Solve Equations (Hyperbolic Substitution):**

\

We use \`{}cosh\`{} and \`{}sinh\`{} as substitutions to simplify equations,
particularly those involving expressions like  \nonconverted{sqrt}
(x{\texttwosuperior} - c{\texttwosuperior}) or  \nonconverted{sqrt}
(x{\texttwosuperior} + c{\texttwosuperior}). This technique is analogous to
trigonometric substitution used in calculus and algebra.

\

**Scenario 1: Equations involving  \nonconverted{sqrt} (x{\texttwosuperior} -
c{\texttwosuperior})**

\

* \ \ **Problem:** Suppose you have an equation with a term like 
\nonconverted{sqrt} (x{\texttwosuperior} - c{\texttwosuperior}), where c is a
constant.

* \ \ **Analogy:** This structure resembles \`{}cosh{\texttwosuperior}(a) - 1
= sinh{\texttwosuperior}(a)\`{}. We want to make the expression under the
square root a perfect square using this identity.

* \ \ **Substitution:** Let **x = c * cosh(a)**. (We need x {\geq} c for this
to work, as cosh(a) {\geq} 1).

* \ \ **Simplification:**

\ \ \ * \ \ x{\texttwosuperior} - c{\texttwosuperior} = (c *
cosh(a)){\texttwosuperior} - c{\texttwosuperior}

\ \ \ * \ \ = c{\texttwosuperior} * cosh{\texttwosuperior}(a) -
c{\texttwosuperior}

\ \ \ * \ \ = c{\texttwosuperior} * (cosh{\texttwosuperior}(a) - 1)

\ \ \ * \ \ = c{\texttwosuperior} * sinh{\texttwosuperior}(a)

\ \ \ * \ \ So,  \nonconverted{sqrt} (x{\texttwosuperior} -
c{\texttwosuperior}) =  \nonconverted{sqrt} (c{\texttwosuperior} *
sinh{\texttwosuperior}(a)) = \textbar c * sinh(a)\textbar . This often
simplifies the original equation significantly by removing the square root.

\

* \ \ **Example:** Solve  \nonconverted{sqrt} (x{\texttwosuperior} - 9) = 4

\ \ \ * \ \ Let x = 3 cosh(a). (Here c = 3)

\ \ \ * \ \ The equation becomes  \nonconverted{sqrt} ((3
cosh(a)){\texttwosuperior} - 9) = 4

\ \ \ * \ \  \nonconverted{sqrt} (9 cosh{\texttwosuperior}(a) - 9) = 4

\ \ \ * \ \  \nonconverted{sqrt} (9 (cosh{\texttwosuperior}(a) - 1)) = 4

\ \ \ * \ \  \nonconverted{sqrt} (9 sinh{\texttwosuperior}(a)) = 4

\ \ \ * \ \ Assume sinh(a) is positive for simplicity here (we might need +/-
depending on context): 3 sinh(a) = 4

\ \ \ * \ \ sinh(a) = 4/3

\ \ \ * \ \ To find 'a', we use the inverse hyperbolic sine: a = arsinh(4/3)
(or sinh⁻{\textonesuperior}(4/3)).

\ \ \ * \ \ Now, substitute back to find x: x = 3 cosh(a) = 3
cosh(arsinh(4/3)).

\ \ \ * \ \ Using the identity cosh{\texttwosuperior}(a) -
sinh{\texttwosuperior}(a) = 1 => cosh{\texttwosuperior}(a) = 1 +
sinh{\texttwosuperior}(a).

\ \ \ * \ \ cosh(a) =  \nonconverted{sqrt} (1 + sinh{\texttwosuperior}(a))
(since cosh(a) is always positive)

\ \ \ * \ \ cosh(a) =  \nonconverted{sqrt} (1 + (4/3){\texttwosuperior}) = 
\nonconverted{sqrt} (1 + 16/9) =  \nonconverted{sqrt} (25/9) = 5/3.

\ \ \ * \ \ Therefore, x = 3 * (5/3) = 5.

\ \ \ * \ \ Check:  \nonconverted{sqrt} (5{\texttwosuperior} - 9) = 
\nonconverted{sqrt} (25 - 9) =  \nonconverted{sqrt} 16 = 4. It works. (Note: x
= -5 also works, which corresponds to the negative sinh case).

\

**Scenario 2: Equations involving  \nonconverted{sqrt} (x{\texttwosuperior} +
c{\texttwosuperior})**

\

* \ \ **Problem:** Suppose you have an equation with a term like 
\nonconverted{sqrt} (x{\texttwosuperior} + c{\texttwosuperior}).

* \ \ **Analogy:** This structure resembles \`{}sinh{\texttwosuperior}(a) + 1
= cosh{\texttwosuperior}(a)\`{}.

* \ \ **Substitution:** Let **x = c * sinh(a)**. (sinh(a) can be any real
number, so x can be any real number).

* \ \ **Simplification:**

\ \ \ * \ \ x{\texttwosuperior} + c{\texttwosuperior} = (c *
sinh(a)){\texttwosuperior} + c{\texttwosuperior}

\ \ \ * \ \ = c{\texttwosuperior} * sinh{\texttwosuperior}(a) +
c{\texttwosuperior}

\ \ \ * \ \ = c{\texttwosuperior} * (sinh{\texttwosuperior}(a) + 1)

\ \ \ * \ \ = c{\texttwosuperior} * cosh{\texttwosuperior}(a)

\ \ \ * \ \ So,  \nonconverted{sqrt} (x{\texttwosuperior} +
c{\texttwosuperior}) =  \nonconverted{sqrt} (c{\texttwosuperior} *
cosh{\texttwosuperior}(a)) = \textbar c * cosh(a)\textbar  = c * cosh(a)
(since c > 0 and cosh(a) > 0).

\

* \ \ **Example:** Solve  \nonconverted{sqrt} (x{\texttwosuperior} + 16) = 5

\ \ \ * \ \ Let x = 4 sinh(a). (Here c = 4)

\ \ \ * \ \ The equation becomes  \nonconverted{sqrt} ((4
sinh(a)){\texttwosuperior} + 16) = 5

\ \ \ * \ \  \nonconverted{sqrt} (16 sinh{\texttwosuperior}(a) + 16) = 5

\ \ \ * \ \  \nonconverted{sqrt} (16 (sinh{\texttwosuperior}(a) + 1)) = 5

\ \ \ * \ \  \nonconverted{sqrt} (16 cosh{\texttwosuperior}(a)) = 5

\ \ \ * \ \ 4 cosh(a) = 5

\ \ \ * \ \ cosh(a) = 5/4

\ \ \ * \ \ a = arcosh(5/4) (or cosh⁻{\textonesuperior}(5/4)). Note: there
are two values of 'a' (positive and negative) that give the same cosh value,
but they lead to opposite sinh values.

\ \ \ * \ \ Now, substitute back to find x: x = 4 sinh(a).

\ \ \ * \ \ Using cosh{\texttwosuperior}(a) - sinh{\texttwosuperior}(a) = 1
=> sinh{\texttwosuperior}(a) = cosh{\texttwosuperior}(a) - 1.

\ \ \ * \ \ sinh{\texttwosuperior}(a) = (5/4){\texttwosuperior} - 1 = 25/16 -
16/16 = 9/16.

\ \ \ * \ \ sinh(a) = {\pm} \nonconverted{sqrt} (9/16) = {\pm}3/4.

\ \ \ * \ \ Therefore, x = 4 * ({\pm}3/4) = {\pm}3.

\ \ \ * \ \ Check:  \nonconverted{sqrt} ( ({\pm}3){\texttwosuperior} + 16) = 
\nonconverted{sqrt} (9 + 16) =  \nonconverted{sqrt} 25 = 5. It works.

\

**Connecting Back to Hypotenuse:**

\

The *reason* this substitution strategy works is the algebraic structure of
the identity \`{}cosh{\texttwosuperior}(a) - sinh{\texttwosuperior}(a) =
1\`{}. This structure is the hyperbolic counterpart to the Pythagorean
identity \`{}cos{\texttwosuperior}({\theta}) + sin{\texttwosuperior}({\theta})
= 1\`{}, which *is* fundamentally derived from the relationship between the
sides (including the hypotenuse) of a right triangle inscribed in a unit
circle. So, while you aren't directly using a hypotenuse *in* the hyperbolic
calculation, you're using an analogous mathematical tool derived from the
geometry of the hyperbola, just as trigonometric substitutions use a tool
derived from the geometry of the circle (and thus, right triangles and
hypotenuses).

\end{document}
